%%%%%%%%%%%%%%%%%%%%%%%%%%%%%%%%%%%%%%%%%

\documentclass[12pt]{article}
\usepackage{amsmath,amsthm}
\usepackage{amssymb}
\usepackage[english]{babel}
\usepackage{latexsym}
\usepackage{amsfonts}
\usepackage{graphicx}
\usepackage{float}
\usepackage{graphics}
\usepackage{epsfig}
\usepackage[T1]{fontenc}
\usepackage[utf8]{inputenc}
\usepackage{hyperref}
\usepackage{authblk}
\usepackage[]{mcode}
\usepackage{natbib}
\usepackage{xcolor} % Required to specify font color
\usepackage{multicol}
\setlength{\columnsep}{0.5cm}
\usepackage{caption}
\usepackage{calc}
\usepackage{longtable}
\usepackage{subcaption}
\usepackage[margin=1in]{geometry}
\usepackage{tikz}
\usepackage{float}
\usepackage{multirow}
\usepackage{graphicx}
\usepackage{hyperref}



\title{VS Code Workshop}
\author{Caroline Welter}

%\renewcommand\Authands{ and }


%\pagenumbering{arabic}

\date{\today}


\definecolor{wvublue}{HTML}{003366}
\definecolor{wvugold}{HTML}{FFCC00}


%----------------------------------------------------------------------------------------------------


% ---------------------------------------------------------------------
\RequirePackage{fancyhdr}  % Needed to define custom headers/footers
\RequirePackage{lastpage}  % Number of pages in the document
\pagestyle{fancy}          % Enables the custom headers/footers
% Headers

\chead{}%
\rhead{ \thepage/\pageref{LastPage}} %Logo MEI to include
% Footers
\lfoot{}%
\cfoot{}%
\rfoot{}%
\renewcommand{\headrulewidth}{0pt}% % No header rule
\renewcommand{\footrulewidth}{0pt}% % No footer rule

\begin{document} 
\maketitle
\pagenumbering{gobble}
\pagenumbering{arabic}
\thispagestyle{empty}


\section{Download Programs}

TEST 

\begin{itemize}
  \item VS Code: https://code.visualstudio.com/download
  \item Anaconda: https://www.anaconda.com/download 
  \item R and RStudio: https://posit.co/download/rstudio-desktop/
  \item Github Desktop: Sign up with your mix account https://github.com/ and download the Desktop version: https://desktop.github.com/
  \item Git: https://git-scm.com/downloads
  \item Git for large files: https://git-lfs.com/
  \item Miktex: https://miktex.org/download 
\end{itemize}

\section{Install VS Code extensions}
\begin{itemize}
  \item Latex Workshop
  \item Code Spell Checker
  \item Grammarly
  \item Python
  \item R
  \item RDebugger
  \item Stata Run
  \item Git Graph
  \item Jupyter
\end{itemize}

\section{Jupiter Notebook: Stata-kernel}

Follow the steps:
\begin{itemize}
  \item \textbf{Stata:}
    \begin{itemize}
      \item[1.] Send a copy to the desktop
      \item[2.] Open properties 
      \item[3.] Change the end of the “target” path to:  /Register
      \item[4.] Run as administrator
    \end{itemize}
\end{itemize}

\begin{itemize}
  \item \textbf{From the Anaconda Prompt run:} 
\end{itemize}

\begin{verbatim}

  conda install -c conda-forge stata_kernel

  python -m stata_kernel.install

\end{verbatim}
  

\begin{itemize}
  \item \textbf{Find config.py and change the stata path:}
\end{itemize}

Find the path to Anaconda, like this one:

\begin{verbatim}
  C:\Users\carol\anaconda3\Lib\site-packages\stata_kernel
\end{verbatim}

  Open config file and change the stata path:

  \begin{verbatim}
    # stata_path = self.get('stata_path', find_path()) # comment this line 

    stata_path = r'C:/Program Files/Stata17/StataSE-64.exe' 
    # include this one, but first check what your stata path is in your 
    laptop and change accordingly

  \end{verbatim}
  
  save and close the file

  
  
\begin{itemize}  
  \item \textbf{In VS Code:} 
    \begin{itemize}
      \item Open the file test.ipynb
      \item Select stata kernel
      \item Clear all outputs
      \item Run code
    \end{itemize}
    \begin{itemize}
      \item Open test2.R 
      \item Run code
    \end{itemize}
\end{itemize}


Example of how to use stata in the Jupiter Notebook:

https://nbviewer.org/github/kylebarron/stata_kernel/blob/master/examples/Example.ipynb


\end{document}